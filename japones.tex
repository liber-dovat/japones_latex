\documentclass[a4paper,12pt,oneside]{report}

%----------------------------------------------
% Preambulo

%\usepackage{colorful}
\usepackage[bw]{colorful}
\usepackage{tikz}

\usepackage{multicol}

% para usar las bullets redondas para marcar los elementos de las listas
% \renewcommand{\labelitemi}{$\bullet$}

\usepackage{CJKutf8}

\usepackage[CJK, overlap]{ruby}
\renewcommand{\rubysep}{-0.2ex}

\def\to{$\longrightarrow$}
\def\then{$\Longrightarrow$}

\def\bv{\textbf{V}} % bold verb
\def\bs{\textbf{S}} % bold sustantive
\def\adj{\textbf{adj}} % bold adjective

\def\vi{$\mathrm{\bv}_1$} % verbo ichi
\def\vn{$\mathrm{\bv}_2$} % verbo ni
\def\vs{$\mathrm{\bv}_3$} % verbo san
\def\vene{$\mathrm{\bv}_\mathrm{\textbf{n}}$} % verbo ene
\def\tenten{。。。}

\newenvironment{changemargin}{%
 \begin{list}{}{%
\setlength{\textwidth}{\paperwidth}
\setlength{\textheight}{\paperheight}
%\setlength{\oddsidemargin}{-1in}
%\setlength{\evensidemargin}{-1in}
% \setlength{\topmargin}{-1in}
% \setlength{\topsep}{-1in}%
   \setlength{\leftmargin}{-0.7in}%
   \setlength{\rightmargin}{0.4in}%
   \setlength{\listparindent}{\parindent}%
   \setlength{\itemindent}{\parindent}%
   \setlength{\parsep}{\parskip}%
 }%
%\centering%
\item[]%
}{\end{list}}

%----------------------------------------------
% Documento

\begin{document}
\begin{CJK*}{UTF8}{min}%[dnp]{JIS}{min}

  %----------------------------------------------
  % Titulo

  %\CJKfamily{goth}
  %\CJKfamily{maru}

  \title{
    \begin{flushright}
      \Huge  日本語\\
      \LARGE \ColCapitulo{Gu\'ia del curso}
    \end{flushright}
  } % \title

  \author{\vspace{7cm}\\
    v0.7\\
    Compilado por Liber Dovat\\
    el \today\\
  } % \author

  \date{}
  \maketitle

  \newpage

  %----------------------------------------------
  %Indice

  \pagenumbering{Roman}
  \tableofcontents
  \thispagestyle{plain}

  \newpage

  %----------------------------------------------
  % Contenido

  \pagenumbering{arabic}

%========================================

%   \chapter{Introducci\'on}
%   \thispagestyle{contenido} % no sacar
%   \pagestyle{contenido}     % no sacar
%
%     れいぞうこ の中に たまご や\ruby{肉}{にく}など が あります。「や、など」
%     \\
%
%     より: Comparación efectiva entre cosas
%     \\
%
%     のほうが: Comparación de gustos
%     \\

%========================================

  \chapter{Japon\'es 2}
  \thispagestyle{contenido} % no sacar
  \pagestyle{contenido}     % no sacar

    \section{Lección 14 - Forma て (て\ruby{形}{けい})}

%         \subsubsection{Grupo 1}
%         \begin{tabular}{ | l | l | }
%           \hline
%           \cellcolor[gray]{0.9} Forma ます & \cellcolor[gray]{0.9} Forma て\\
%           \hline
%           きます & きいて\\
%           \hline
%         \end{tabular}

        \subsubsection{Grupo 1 - Serie い}
        \begin{tabular}{rcr}
          か\textbf{き}ます & \then & か\fbox{\textbf{いて}} \\
          $*$い\textbf{き}ます & \then & い\fbox{\textbf{って}} \\
          いそ\textbf{ぎ}ます & \then & いそ\fbox{\textbf{いで}} \\
          &&\\
          の\textbf{み}ます & \then & の\fbox{\textbf{んで}} \\
          よ\textbf{び}ます & \then & よ\fbox{\textbf{んで}} \\
          し\textbf{に}ます & \then & し\fbox{\textbf{んで}} \\
          と\textbf{り}ます & \then & と\fbox{\textbf{って}} \\
          あ\textbf{い}ます & \then & あ\fbox{\textbf{って}} \\
          ま\textbf{ち}ます & \then & ま\fbox{\textbf{って}} \\
          け\textbf{し}ます & \then & け\fbox{\textbf{して}} \\
        \end{tabular}

        \subsubsection{Grupo 2 - Serie え}
        \begin{tabular}{rcr}
          たべます & \then & たべ\textbf{て} \\
          ねます & \then & ね\textbf{て} \\
          &&\\
          $*$みます & \then & み\textbf{て} \\
          $*$います & \then & い\textbf{て} \\
          $*$おきます & \then & おき\textbf{て} \\
          $*$かります & \then & かり\textbf{て} \\
          $*$おります & \then & おり\textbf{て} \\
          $*$あびます & \then & あび\textbf{て} \\
        \end{tabular}

        \subsubsection{Grupo 3 - します y きます}
        \begin{tabular}{rcr}
          します & \then & し\textbf{て} \\
          きます & \then & き\textbf{て} \\
        \end{tabular}

      \subsubsection{Solicitar, dar instrucciones o consejos:}
        \fbox{\bv(て) + ください}

        \begin{itemize}
          \item すみませんが、かんじの よみ方を おひえてください。\\
                Disculpe, >Podr\'ia enseñarme c\'omo leer kanji? (solicitud)
          \item ぜひ あそびに \ruby{来}{き}て ください。\\
                No deje de visitarnos. (invitaci\'on)
          \item ここに でんわばんごうと なまえを かいて ください。\\
                Escriba aqu\'i su n\'umero de tel\'efono y su nombre, por favor. (instrucci\'on)
        \end{itemize}

      \subsubsection{Conjugaci\'on, progreso de la acci\'on:}
        \fbox{\vi(て) + います}

        \begin{itemize}
          \item マリアさん は パブロさんと はなして います。\\
                Mar\'ia est\'a hablando con Pablo.
          \item
            \begin{itemize}
              \item[-] 今 雨が ふって いますか。\\
                       >Ahora est\'a lloviendo?
              \item[-] はい、ふって います。\\
                       S\'i, est\'a lloviendo.
            \end{itemize}
        \end{itemize}

      \subsubsection{Ofrecerce para realizar una acci\'on:}
        \fbox{\bv ましょうか}

        \begin{itemize}
          \item
            \begin{itemize}
              \item[-] あついですね、まど を あけましょうか。\\
                       Hace calor, >no? >Abro la ventana?
              \item[-] はい、おねがいします。\\
                       S\'i, por favor.
            \end{itemize}
          \item
            \begin{itemize}
              \item[-] にもつ を もちましょうか。\\
                       >Le llevo la valija?
              \item[-] いいえ、けっこうです。\\
                       No, est\'a bien.
            \end{itemize}
        \end{itemize}

      \subsubsection{Realizar dos acciones simultaneamente:}
        \fbox{\vi(ます) + ながら、\vn}

        \begin{itemize}
          \item \ruby{音楽}{おんがく} を ききながら \ruby{食事}{しょくじ}します。\\
                Como escuchando m\'usica.
          \item コーヒー を おみながら べんきょうします。\\
                Estudio tomando caf\'e.
        \end{itemize}
        \hfill

        Tambi\'en se utiliza para acciones que contin\'uan permanentemente

          \begin{itemize}
            \item はたらきながら 日本語 を べんきょうします。\\
                  Estoy trabajando y estudiando japon\'es.
          \end{itemize}

%========================================

    \section{Lección 15}

      \subsubsection{Se puede; dar permiso:}
        \fbox{\bv(て) も いいです}

        \subparagraph{Permiso:}

        \begin{itemize}
          \item ノート を 見てもいいです。\\
                Se puede mirar el cuaderno.
        \end{itemize}

        \subparagraph{Pedir permiso:}

        \begin{itemize}
          \item
            \begin{itemize}
              \item[-] たばこ を すっても いいですか。\\
                       >Se puede fumar?
              \item[-] いいえ、だめです。ここは きんえんです。\\
                       No, no se puede. Aqu\'i est\'a prohibido.
            \end{itemize}
          \item
            \begin{itemize}
              \item[-] この せき に すわっても いいですか。\\
                       >Me puedo sentar en esta silla?
              \item[-] ええ、いいですよ。どうぞ。\\
                       Si, est\'a bien. Adelante.
              \item[-] \ruby{住}{す}みません。ちょっと。\\
                       Lo siento, pero \ldots
            \end{itemize}
        \end{itemize}

      \subsubsection{No se puede:}
        \fbox{\bv(て) は いけません}

        \begin{itemize}
          \item バス の 中で たばこ を すわって は いけません。\\
                Est\'a prohibido fumar dentro del autob\'us.
          \item ここに 車 を 止めて は いけません。\\
                Est\'a prohibido estacionar aqu\'i.
        \end{itemize}

      \subsubsection{Acci\'on cont\'inua en el tiempo:}
        O resultado de una acci\'on pasada:\\

        \fbox{\bv(て) います}

        \begin{itemize}
          \item \ruby{木村}{きむら}先生は けっこんしています。\\
                La profesora Kimura est\'a casada.
          \item 私は 田中さん を しっています。\\
                Conozco al Sr. Tanaka.
          \item パブロさん は カネローネス に 住んでいます。\\
                Pablo vive en Canelones.
          \item 私は カメラ を 持っていあす。\\
                Tengo una c\'amara.
          \item マリアさん は すわっています。\\
                Mar\'ia est\'a sentada.
          \item まど は あいています。\\
                La ventana est\'a abierta.
        \end{itemize}

        \subparagraph{Atenci\'on:}
          La negaci\'on del verbo 「しっています」 es 「しりません」.\\

        En $1)$ vemos un ejemplo de acci\'on cont\'inua en el tiempo, y en $2)$ una acci\'on que sucedi\'o en el pasado pero a\'un perduran sus resultados:

        \begin{enumerate}
          \item 本 を よんでいます。\\
                Leo un libro.
          \item けっこん しています。\\
                Est\'a casado.
        \end{enumerate}

      \subsubsection{Acci\'ones rutinarias o de larga duraci\'on:}
        \fbox{\bv(て) います}

        \begin{itemize}
          \item スーパー で せっけん を うっています。\\
                Se vende jab\'on en el supermercado.
          \item \ruby{松本}{まつもと}さん は ABCで はたらいています。\\
                El Sr.Matsumoto trabaja en la empresa ABC.
          \item まいあさ ジョギング を しています。\\
                Corro todas las mañanas.
        \end{itemize}
        \hfill

        Las acciones rutinarias del pasado se expresan con \bv(て) いました.

        \begin{itemize}
          \item まえに テニス を していましたが、今 は していません。\\
                Antes jugaba al tenis pero ahora no juego.
        \end{itemize}

      \subsubsection{Aprobar lo dicho por el hablante:}
        \fbox{「はい」と「いいえ」}

        \begin{itemize}
          \item
            \begin{itemize}
              \item[-] さしみ は 食べませんか。\\
                       >No come Sashimi?
              \item[-] はい、食べません。\\
                       No, no como.
              \item[-] いいえ、食べます。\\
                       S\'i, como.
            \end{itemize}
          \item
            \begin{itemize}
              \item[-] その 肉 は おいしくないですか。\\
                       >Esa carne no es rica?
              \item[-] ええ、あまり おいしくないです。\\
                       No, no es rica.
              \item[-] いいえ、おいしいです。\\
                       S\'i, es rica.
            \end{itemize}
        \end{itemize}

%========================================

    \section{Lección 16}

      \subsubsection{Secuencial, dos o mas verbos ocurren en secuencia:}
        \fbox{\vi(て)、\vn(て)、\tenten、\vene(ます)}

        \begin{itemize}
          \item カルロすさん は あさ、ジョギング を して、シャワー を あびて、大学へ行きます。\\
                Carlos corre por la mañana, se da un baño, y va a la facultad.
        \end{itemize}

        \subparagraph{Atenci\'on:}
          Las acciones deben estar en un orden coherente.

      \subsubsection{Acciones que se realizan luego de concluir otra:}
        \fbox{\vi(て)から、\vn(ます)}

        \begin{itemize}
          \item ふるさと へ かえって から、父の\ruby{会社}{かいしゃ}で はたらきます。\\
                Despu\'es de volver a mi pueblo natal, voy a trabajar en la empresa de mi padre.
          \item えいが が おわって から、レストランで\ruby{食事}{しょくじ}しました。\\
                Despu\'es de terminar la pel\'icula, fui a comer a un restor\'an.
        \end{itemize}

      \subsubsection{Conectar oraciones que usan い-adjetivos:}
        \fbox{\adj-い(〜い×) \to 〜くて、\tenten}

        \begin{itemize}
          \item マリアさん は わかくて、\ruby{元気}{げんき}です。\\
                Mar\'ia es joven y saludable.
          \item きのう は てんきが よくて、すこし あつかったです。\\
                Ayer hixo buen clima y estuvo un poco caluroso.
        \end{itemize}

      \subsubsection{Conectar oraciones que usan な-adjetivos o sustantivos:}
        \fbox{\bs/\adj-な(な×) \to で}

        \begin{itemize}
          \item パブロさん は ハンサムで、しせつです。\\
                Pablo es apuesto y amable.
          \item サンパウロ は にぎやかで、おもしとい です。\\
                San Pablo es movida e interesante.
        \end{itemize}

      \subsubsection{Causa o motivo:}
        \fbox{\vi(て)、\tenten、\vene(ます)}

        \begin{itemize}
          \item 風を ひいて、じゅぎょう を 休みました。\\
                Como estoy resfriado, falt\'e a clase.
          \item しゅくだい が たくさん あって、いぞがしいです。\\
                Como tengo muchos deberes, estoy ocupado.
        \end{itemize}

%========================================

    \section{Lección 17 - Forma ない (ない\ruby{形}{けい})}

%         \subsubsection{Grupo 1}
%         \begin{tabular}{ | l | l | }
%           \hline
%           \cellcolor[gray]{0.9} Forma ます & \cellcolor[gray]{0.9} Forma ない\\
%           \hline
%           きます & こない\\
%           \hline
%         \end{tabular}

        \subsubsection{Grupo 1 - Serie い}
        \begin{tabular}{rcr}
          か\textbf{き}ます & \then & か\textbf{か}ない \\
          い\textbf{き}ます & \then & い\textbf{か}ない \\
          いそ\textbf{ぎ}ます & \then & いそ\textbf{が}ない \\
          の\textbf{み}ます & \then & の\textbf{ま}ない \\
          よ\textbf{び}ます & \then & よ\textbf{ば}ない \\
          し\textbf{に}ます & \then & し\textbf{な}ない \\
          と\textbf{り}ます & \then & と\textbf{ら}ない \\
          $*$あ\textbf{い}ます & \then & あ\textbf{わ}ない \\
          ま\textbf{ち}ます & \then & ま\textbf{た}ない \\
          け\textbf{し}ます & \then & け\textbf{さ}ない \\
        \end{tabular}

        \subsubsection{Grupo 2 - Serie え}
        \begin{tabular}{rcr}
          たべます & \then & たべない \\
          ねます & \then & ねない \\
          &&\\
          $*$みます & \then & みない \\
          $*$います & \then & いない \\
          $*$おきます & \then & おきない \\
          $*$かります & \then & かりない \\
          $*$おります & \then & おりない \\
          $*$あびます & \then & あびない \\
        \end{tabular}

        \subsubsection{Grupo 3 - します y きます}
        \begin{tabular}{rcr}
          します    & \then & しない \\
          $*$きます & \then & \textbf{こ}ない \\
        \end{tabular}

      \subsubsection{Indicar que no haga algo:}
        \fbox{\bv(ない) + ないで ください}

        \begin{itemize}
          \item ここ で しゃしん を とらない で ください。\\
                No saque fotos aquí.
          \item かさ を わすれないで ください。\\
                No se olvide del paraguas.
          \item あまり たばこ を すわないで ください。\\
                No fume tanto.
        \end{itemize}

      \subsubsection{Algo debe hacerse, por obligación o sin la voluntad:}
        \fbox{\bv(ない) + なければ なりません}

        \begin{itemize}
          \item しゅくだい を しなければ なりません。\\
                Debo hacer los deberes.
          \item 日本の\ruby{家}{うち}で くつ を ぬがなければ なりません。\\
                En las casas japonesas se debe quitar los zapatos.
        \end{itemize}

      \subsubsection{No es necesario hacer algo:}
        \fbox{\bv(ない) + なくても いいです}

        \begin{itemize}
          \item お金を はわらなくても いいです。\\
                No es necesario que pague.
        \end{itemize}

      \subsubsection{Medios o situación al realizar una acción:}
        \fbox{\vi(て)、\vn}

        \begin{itemize}
          \item コーヒー に さとう を 入れて \ruby{飲}{の}みます。\\
                Tomo caf\'e con azúcar.
        \item 私は いつも たって アイロン を かけます。\\
                Siempre plancho parado.
        \end{itemize}

      \subsubsection{Medios o situación al realizar una acción con forma negativa:}
        \fbox{\vi(ない) + ないで、\vn}

        \begin{itemize}
          \item は を みがかないで ねました。\\
                Me acost\'e sin lavarme los dientes.
          \item さとう 入れないで、コーヒー を \ruby{飲}{の}みました。\\
                Tom\'e el caf\'e sin az\'ucar.
        \end{itemize}

%========================================

    \section{Lección 18 - Forma じしょ(辞書\ruby{形}{けい})}

%         \subsubsection{Grupo 1}
%         \begin{tabular}{ | l | l | }
%           \hline
%           \cellcolor[gray]{0.9} Forma ます & \cellcolor[gray]{0.9} Forma じしょ\\
%           \hline
%           きます & くる\\
%           \hline
%         \end{tabular}

      \subsubsection{Grupo 1 - Serie い}
        \begin{tabular}{rcr}
          か\textbf{き}ます & \then & かく \\
          い\textbf{き}ます & \then & いく \\
          いそ\textbf{ぎ}ます & \then & いそぐ \\
          の\textbf{み}ます & \then & のむ \\
          よ\textbf{び}ます & \then & よぶ \\
          し\textbf{に}ます & \then & しぬ \\
          ま\textbf{ち}ます & \then & まつ \\
          け\textbf{し}ます & \then & けす \\
        \end{tabular}

      \subsubsection{Grupo 2 - Serie え}
        \begin{tabular}{rcr}
          たべます & \then & たべる \\
          ねます & \then & ねる \\
          &&\\
          $*$みます & \then & みる \\
          $*$います & \then & いない \\
          $*$おきます & \then & おきない \\
          $*$かります & \then & かりる \\
          $*$おります & \then & おりる \\
          $*$あびます & \then & あびる \\
        \end{tabular}

      \subsubsection{Grupo 3 - します y きます}
        \begin{tabular}{rcr}
          します & \then & する \\
          きます & \then & くる \\
        \end{tabular}

      \subsubsection{Verbo como sustantivo:}
        \fbox{\bv(じしょ) + こと}

      \subsubsection{Pasatiempo:}
        \fbox{しゅみ は \bv(じしょ) + ことです}

        \begin{itemize}
          \item 私の しゅみ は \ruby{音楽}{おんがく}です。\\
                Mi pasatiempo es la m\'usica.
          \item 私の しゅみ は \ruby{音楽}{おんがく} を きく ことです。\\
                Mi pasatiempo es escuchar m\'usica.
          \item しゅみ は \ruby{音楽}{おんがく} を えんそうする ことです。\\
                Mi pasatiempo es tocar m\'usica.
        \end{itemize}

      \subsubsection{Gustos:}
        \fbox{\bs は \bv(じしょ) + ことが すきです}

        \begin{itemize}
          \item 私は サッカーが すきです。\\
                Me gusta el f\'utbol.
          \item 私は サッカーを 見ることが すきです。\\
                Me gusta mirar f\'utbol.
        \end{itemize}

        \subparagraph{Atenci\'on:}
          Se puede usar「の」en lugar de「こと」para hacer la frase un poco informal:

        \begin{itemize}
          \item サッカーを する のが すきです。\\
                Me gusta jugar al f\'utbol.
          \item え を かく のが すきです。\\
                Me gusta pintar.
        \end{itemize}

      \subsubsection{Ser capaz; es posible:}
        \fbox{\bs/\bv(じしょ) こと が できます}

        \subparagraph{Con sustantivo:}

        \begin{itemize}
          \item マリアさん は 日本語 が できます。\\
                Mar\'ia sabe japon\'es.
          \item \ruby{雪}{ゆき}が たくさん ふりました から、ことし は スキー が できます。\\
                Como nev\'o mucho, este año se puede esquiar.
        \end{itemize}

        \subparagraph{Con verbo:}

        \begin{itemize}
          \item マリアさん は かんじ を よむこと が できます。\\
                Mar\'ia sabe leer kanji.
          \item としょかんで 本を かりることが できまう。\\
                Se puede pedir libros en la biblioteca.
        \end{itemize}

        \subparagraph{Atenci\'on:}
          Se puede usar el verbo {できます} para indicar una acci\'on que se cumple y concluye, indicando el lugar con la part\'icula に.

        \begin{itemize}
          \item えきの\ruby{前}{まえ}に大きいスーパーができました。\\
                Hicieron un supermercado grande frente a la estaci\'on de trenes.
          \item
            \begin{itemize}
              \item[-] とけい の しゅうり は いつ できますか。\\
                       >Cu\'ando terminar\'a la reparaci\'on del reloj?
              \item[-] あした できます。\\
                       La terminar\'e mañana.
            \end{itemize}
        \end{itemize}

      \subsubsection{Un acontecimiento ocurre antes que otro:}
        \fbox{\bsの/\vi(じしょ)/cuantificador de tiempo まえに、\vn}

        \subparagraph{Con verbo:}

        El tiempo verbal se indica al final de la oraci\'on:

        \begin{itemize}
          \item ウルグアイに来る前にスペイン語をべんきょうしました。\\
                Antes de venir a uruguay estudi\'e español.
          \item ねる前に本をよみます。\\
                Antes de dormir leo un libro.
        \end{itemize}

        \subparagraph{Con sustantivo:}

        Se le agrega la part\'icula {の} al sustantivo:

        \begin{itemize}
          \item \ruby{食事}{しょくじ}の前に手をあらいます。\\
                Antes de comer me lavo las manos.
        \end{itemize}

        \subparagraph{Con cuantificador (per\'iodo de tiempo):}

        Se le agrega el cuantificador y no se usa の:

        \begin{itemize}
          \item 田中さんはー時間まえに出かけました。\\
                El Sr.Tanaka sali\'o hace una hora.
        \end{itemize}

%========================================

\chapter{Resumen}
  \thispagestyle{contenido} % no sacar
  \pagestyle{contenido}     % no sacar

  \begin{itemize}
    \item Solicitar, dar instrucciones o consejos:\\
          \bv(て)ください
    \item Conjugaci\'on, progreso de la acci\'on:\\
          \vi(て)います
    \item Ofrecerce para realizar una acci\'on:\\
          \bvましょうか
    \item Realizar dos acciones simultaneamente:\\
          \vi(ます)ながら、\vn
    \item Se puede; dar permiso:\\
          \bv(て)もいいです
    \item No se puede:\\
          \bv(て)はいけません
    \item Acci\'ones rutinarias o de larga duraci\'on:\\
          \bv(て) います
    \item Aprobar lo dicho por el hablante:\\
          「はい」と「いいえ」
    \item Secuencial, dos o mas verbos ocurren en secuencia:\\
          \vi(て)、\vn(て)、\tenten、\vene(ます)
    \item Acciones que se realizan luego de concluir otra:\\
          \vi(て)から、\vn(ます)
    \item Conectar oraciones que usan い-adjetivos:\\
          \adj-い(〜い×)\to〜くて、\tenten
    \item Conectar oraciones que usan な-adjetivos o sustantivos:\\
          \bs/\adj-な(な×)\toで
    \item Causa o motivo:\\
          \vi(て)、\tenten、\vene(ます)
    \item Indicar que no haga algo:\\
          \bv(ない)ないでください
    \item Algo debe hacerse, por obligación o sin la voluntad:\\
          \bv(ない)なければなりません
    \item No es necesario hacer algo:\\
          \bv(ない)なくてもいいです
    \item Medios o situación al realizar una acción:\\
          \vi(て)、\vn
    \item Medios o situación al realizar una acción con forma negativa:\\
          \vi(ない)ないで、\vn
    \item Verbo como sustantivo:\\
          \bv(じしょ)こと
    \item Pasatiempo: (1)\\
          しゅみは\bv(じしょ)ことです
    \item Pasatiempo: (2)\\
          しゅみは\bv(じしょ)ことです
    \item Gustos:\\
          \bsは\bv(じしょ)ことがすきです
    \item Ser capaz; es posible:\\
          \bs/\bv(じしょ)ことができます
    \item Un acontecimiento ocurre antes que otro:\\
          \bsの/\vi(じしょ)/cuantificador de tiempoまえに、\vn
  \end{itemize}

%========================================

\chapter{Kanji}
  \thispagestyle{contenido} % no sacar
  \pagestyle{contenido}     % no sacar

\begin{changemargin}

\begin{multicols}{2}

%   \begin{tabular}{rl}
%     \fbox{\Huge 月}
%     &
%     \begin{tabular}{l}
%       ゲツ, ガツ \\
%       mes, luna \\
%       \\
%       つき \\
%       mes, luna
%     \end{tabular}
%   \end{tabular}
%
%   \begin{tabular}{rl}
%     \fbox{\Huge 金}
%     &
%     \begin{tabular}{l}
%       キン, コン \\
%       oro \\
%       \\
%       かね, かな \\
%       oro, dinero, metal \\
%     \end{tabular}
%   \end{tabular}
%
%   \begin{tabular}{rl}
%     \fbox{\Huge 食}
%     &
%     \begin{tabular}{l}
%       ショク, 「ジキ」 \\
%       alimento, comida, eclipse\\
%       \\
%       く.う, 「く.らう」, た.べる \\
%       comer
%     \end{tabular}
%   \end{tabular}

  \begin{minipage}{1.1in}
    \begin{tikzpicture}
      \node[scale=5] {月};
    \end{tikzpicture}
  \end{minipage}
  \begin{minipage}{2in}
    \begin{tabular}{l}
      ゲツ, ガツ \\
      mes, luna \\
      \\
      つき \\
      mes, luna
    \end{tabular}
  \end{minipage}

  \begin{minipage}{1.1in}
    \begin{tikzpicture}
      \node[scale=5] {金};
    \end{tikzpicture}
  \end{minipage}
  \begin{minipage}{2in}
    \begin{tabular}{l}
      キン, コン \\
      oro \\
      \\
      かね, かな \\
      oro, dinero, metal \\
    \end{tabular}
  \end{minipage}

  \begin{minipage}{1.1in}
    \begin{tikzpicture}
      \node[scale=5] {食};
    \end{tikzpicture}
  \end{minipage}
  \begin{minipage}{2in}
    \begin{tabular}{l}
      ショク, 「ジキ」 \\
      alimento, comida, eclipse\\
      \\
      く.う, 「く.らう」, た.べる \\
      comer
    \end{tabular}
  \end{minipage}

\end{multicols}
\end{changemargin}

  %============================================== Bibliografia

%   \begin{thebibliography}{20}
%
%     \thispagestyle{biblio} % no sacar
%     \pagestyle{biblio}     % no sacar
%
%     \bibitem{maggiolo}
%     Oscar Maggiolo,
%     Cap\'itulo \emph{Pol\'itica de desarrollo cient\'ifico y tecnol\'ogico de Am\'erica Latina},
%     perteneciente al texto \emph{Reflexiones sobre la investigaci\'on cient\'ifica}.
%
%   \end{thebibliography}

  %============================================== Bibliografia - End

\end{CJK*}
\end{document}
