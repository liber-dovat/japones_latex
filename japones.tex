\documentclass[a4paper,12pt,oneside]{report}

%----------------------------------------------
% Preambulo

\usepackage{colorful_color}
\usepackage{tikz}
\usepackage{CJKutf8}

\usepackage[CJK, overlap]{ruby}
\renewcommand{\rubysep}{-0.2ex}

\def\then{\Longrightarrow}

\def\vb{\textbf{V}}       % verbo bold

\def\vi{$\mathrm{\vb}_1$} % verbo ichi
\def\vn{$\mathrm{\vb}_2$} % verbo ni
\def\vs{$\mathrm{\vb}_3$} % verbo san

%----------------------------------------------
% Documento

\begin{document}
\begin{CJK*}{UTF8}{min}%[dnp]{JIS}{min}

  %----------------------------------------------
  % Titulo

  %\CJKfamily{goth}
  %\CJKfamily{maru}

  \title{
    \begin{flushright}
      \Huge  日本語\\
      \LARGE \ColDarkBlue{Guía del curso}
    \end{flushright}
  } % \title

  \author{\vspace{7cm}\\
    v0.3\\
    Compilado por Liber Dovat\\
    el \today\\
  } % \author

  \date{}
  \maketitle

  \newpage

  %----------------------------------------------
  %Indice

  \pagenumbering{Roman}
  \tableofcontents
  \thispagestyle{plain}

  \newpage

  %----------------------------------------------
  % Contenido

  \pagenumbering{arabic}

  %============================================== Capitulo 1

  \chapter{Introducci\'on}
  \thispagestyle{contenido} % no sacar
  \pagestyle{contenido}     % no sacar

    れいぞうこ の中に たまご や\ruby{肉}{にく}など が あります。「や、など」
    \\

    より: Comparación efectiva entre cosas
    \\

    のほうが: Comparación de gustos
    \\

  %============================================== Capitulo 1 - End

  %============================================== Capitulo 2

  \chapter{Japon\'es 2}
  \thispagestyle{contenido} % no sacar
  \pagestyle{contenido}     % no sacar

    \section{Lección 16}

      \subsubsection{Secuencial, dos o mas verbos ocurren en secuencia:}
      \fbox{\vi(て)、\vn(て)、。。。、\vs(ます)}
      \\

      カルロすさん は あさ、ジョギング を して、シャワー を あびて、大学へ行きます。\\
      \indent Carlos corre por la mañana, se da un baño, y va a la facultad.
      \\

      \subsubsection{Acciones que se realizan luego de concluir otra:}
      \fbox{\vi(て)から、\vn(ます)}
      \\

      ふるさと へ かえって から、父の\ruby{会社}{かいしゃ}で はたらきます。\\
      \indent Después de volver a mi pueblo natal, voy a trabajar en la empresa de mi padre.
      \\

      えいが が おわって から、レストランで\ruby{食事}{しょくじ}しました。\\
      \indent Después de terminar la película, fui a comer a un restorán.

    \section{Lección 17}

      \subsubsection{Indicar que no haga algo:}
      \fbox{\vb(ない) + ないで ください}
      \\

      ここ で しゃしん を とらない で ください。\\
      \indent No saque fotos aquí.

      \subsubsection{Algo debe hacerse, por obligación o sin la voluntad:}
      \fbox{\vb(ない) + なければ なりません}
      \\

      しゅくだい を しなければ なりません。\\
      \indent Debo hacer los deberes.

      \subsubsection{No es necesario hacer algo:}
      \fbox{\vb(ない) + なくても いいです}
      \\

      お金を はわらなくても いいです。\\
      \indent No es necesario que pague.

      \subsubsection{Medios o situación al realizar una acción:}
      \fbox{\vi(て)、\vn}
      \\

      コーヒー に さとう を 入れて \ruby{飲}{の}みます。\\
      \indent Tomo caf\'e con azúcar.
    \\

      私は いつも たって アイロン を かけます。\\
      \indent Siempre plancho parado.

      \subsubsection{Medios o situación al realizar una acción con forma negativa:}
      \fbox{\vi(ない) + ないで、\vn}
      \\

      は を みがかないで ねました。\\
      \indent Me acost\'e sin lavarme los dientes.
      \\

      さとう 入れないで、コーヒー を \ruby{飲}{の}みました。\\
      \indent Tom\'e el caf\'e sin az\'ucar.


  %============================================== Capitulo 2 - End

  %============================================== Capitulo Kanji

  \chapter{Kanji}
  \thispagestyle{contenido} % no sacar
  \pagestyle{contenido}     % no sacar

  \begin{minipage}{1.2in}
      \begin{tikzpicture}
        \node[scale=5] {月};
      \end{tikzpicture}
  \end{minipage}
  \begin{minipage}{3in}
    \begin{tabular}{l}
      ゲツ, ガツ \\
      mes, luna \\
      \\
      つき \\
      mes, luna
    \end{tabular}
  \end{minipage}

  \begin{minipage}{1.2in}
      \begin{tikzpicture}
        \node[scale=5] {金};
      \end{tikzpicture}
  \end{minipage}
  \begin{minipage}{3in}
    \begin{tabular}{l}
      キン, コン \\
      oro \\
      \\
      かね, かな \\
      oro, dinero, metal \\
    \end{tabular}
  \end{minipage}

  \begin{minipage}{1.2in}
      \begin{tikzpicture}
        \node[scale=5] {食};
      \end{tikzpicture}
  \end{minipage}
  \begin{minipage}{3in}
    \begin{tabular}{l}
      ショク, 「ジキ」 \\
      alimento, comida, eclipse\\
      \\
      く.う, 「く.らう」, た.べる \\
      comer
    \end{tabular}
  \end{minipage}

% \begin{table}[ht]
% \begin{minipage}[b]{0.5\linewidth}\centering
%   \begin{tabular}{|c|c|c|}
%   \hline
%   1&1&1\\
%   \hline
%   2&2&2\\
%   \hline
%   \end{tabular}
% \end{minipage}
% \hspace{0.5cm}
% \begin{minipage}[b]{0.5\linewidth}
% \centering
% \begin{tabular}{|c|c|c|}
% \hline
% 1&1&1\\
% \hline
% 2&2&2\\
% \hline
% \end{tabular}
% \end{minipage}
% \end{table}

  %============================================== Bibliografia

%   \begin{thebibliography}{20}
%
%     \thispagestyle{biblio} % no sacar
%     \pagestyle{biblio}     % no sacar
%
%     \bibitem{maggiolo}
%     Oscar Maggiolo,
%     Cap\'itulo \emph{Pol\'itica de desarrollo cient\'ifico y tecnol\'ogico de Am\'erica Latina},
%     perteneciente al texto \emph{Reflexiones sobre la investigaci\'on cient\'ifica}.
%
%   \end{thebibliography}

  %============================================== Bibliografia - End

\end{CJK*}
\end{document}
